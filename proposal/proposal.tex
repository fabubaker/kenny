\documentclass{proc}

\title{Slipstream: Adaptive Co-Scheduling for Heterogenous Web Workloads}
\author{Fadhil Abubaker}
\date{}

\begin{document}
\maketitle

\section{Motivation}

Present-day web applications process thousands of requests per second, each with
strict response times. Studies have shown that the average user will abandon a
web page if the load time is more than 3 seconds. Hence, web applications
typically use background jobs to process long-running requests, as well as tasks
that are non-user-facing. Executing background jobs involves a message queue for
queuing jobs, and a worker pool for processing them. Workers are distributed
across multiple nodes to support high-load scenarios. Figure 1 shows an example
of a web application that uses Celery, a popular message queue for queuing jobs,
with Celery workers running on multiple nodes.

An optimization metric in such a scenario is job throughput, defined as the
number of jobs processed per unit of time. Increasing job throughput typically
equates to spawning more workers, which in turn increases the number of jobs
that can be processed. However, care must be taken to spawn the right number of
workers on the given hardware. Eagerly spawning workers can lead to
over-utilization of hardware resources, which can worsen job completion times
and throughput. On the other hand, conservatively spawning workers can lead to
under-utilization, leaving performance improvements on the table.

The optimal number of workers to spawn is a function of workload
characteristics. If incoming jobs are CPU-bound, then the number of workers to
spawn per node is equal to the number of CPU cores on the node\footnote{This
assumes that jobs are single-threaded. This is a fair assumption, since most web
applications are written using frameworks in interpreted languages such as
Python and Ruby that use a global interpreter lock (GIL).}. If incoming jobs are
I/O-bound, then the number of workers to spawn per node is correlated with the
capacity of the I/O sub-system. Figure 2 shows throughput degradation on a node
with four cores as a function of the number of workers, for both types of
workloads.

In practice, however, web workloads are heterogenous: they involve a mix of both
CPU and I/O-bound jobs. For example, an application could have jobs for training
models for learning user preferences (CPU-bound), as well as jobs that aggregate
data from multiple APIs (I/O-bound). One observation for maximizing the
throughput of such workloads is that CPU-bound jobs can be co-scheduled with
I/O-bound jobs, since they use complementary resources.

Using this observation, we build Slipstream, a scheduler for web workloads that
can adaptively co-schedule jobs to workers. Slipstream consists of two
components:
\begin{enumerate}
  \item An estimation model that can quantify the CPU and I/O characteristics of
  incoming jobs.
  \item A scheduling policy that uses the above estimation to spawn the optimal
  number of workers and assign jobs to them.
\end{enumerate}

\section{Methodology}

We will build Slipstream on top of RQ, a Python library that uses Redis for
queueing jobs and executing them on workers. Slipstream will run as an
independent service that monitors the job queue, spawns workers, assigns jobs
and profiles job execution to learn the characteristics of its workload. At
first, a conservative number of workers will be spawned. Gradually, as the
system learns the workload characteristics of each job, more workers will be
spawned and jobs will be assigned to them using a workload-aware scheduling policy.

To evaluate Slipstream, we will use jobs with varying inputs that emulate the
mixed workloads seen in a typical web application such as machine-learning and
data-gathering jobs. We then compare it against static scheduling policies that
allocate a fixed number of workers to each node. The expectation is that
Slipstream will exhibit greater throughput due to more effective utilization of
resources.

\end{document}
