\documentclass[draft]{proc}

\usepackage[backend=biber, style=trad-abbrv]{biblatex}
\usepackage[margin=1in, includeheadfoot]{geometry}
\usepackage{libertine}
\usepackage{titling}

\setlength{\droptitle}{-5em}
\pagestyle{plain}

\addbibresource{references.bib}

\title{High Availability For Key-Value Stores Using Checkpoint/Restore}
\author{Fadhil Abubaker, Hussain Sadiq Abuwala}
\date{}

\begin{document}
\maketitle

\section{Introduction}

High availability (HA) is an important requirement for modern distributed
systems. HA is achieved through replication, where multiple replicas of the
system are run on different nodes for redundancy. Updates made to one replica
are propagated to the others in a synchronous or asynchronous manner.
Additionally, replication can be active-active, where any replica can accept
updates or active-passive, where only a master replica can accept updates
\cite{Dangers}. Thus, the failure of one replica does not affect the operation
of the entire system, as another replica can take its place, and continue
serving requests.

However, implementing HA within a distributed system is a challenging task. As
mentioned in \cite{RemusDB}, to build a simple active-standby replication into a
DBMS, the system has to implement propagating updates from the active replica to
the standby, coordinate transactions between the replicas and ensure atomic
handover from active to standby in the face of a failure. Moreover, these
components have to be carefully implemented so as to have minimal impact on the
performance of the underlying system.

Given the complexity of implementing HA, the question arises whether it should
be pushed outside of the system. One such approach that has been studied is
virtual machine (VM) replication, where changes made to a primary VM are
propagated to a secondary VM \cite{Hypervisor, Remus, Scales2010TheDA}. Prior
work has also looked at how to adapt datastores to run on such VMs to guarantee
HA \cite{RemusDB}.

In this paper, we present a HA technique for key-value stores using
container-based checkpoint/restore. Our approach is similar to VM replication,
except the granularity of replication is at the container-level, rather than the
VM-level. With the widespread adoption of containers for modern-day computing,
this technique can be applied to any datastore running in the cloud.

\section{System Architecture}

\section{Evaluation}

\section{Conclusion}

\printbibliography

\end{document}
