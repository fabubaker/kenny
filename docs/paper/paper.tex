\documentclass[draft]{proc}

\usepackage[backend=biber, style=trad-abbrv]{biblatex}
\usepackage[margin=1in, includeheadfoot]{geometry}
\usepackage{libertine}
\usepackage{titling}

\newcommand{\bit}{\begin{itemize}}
\newcommand{\eit}{\end{itemize}}

\setlength{\droptitle}{-5em}
\pagestyle{plain}

\addbibresource{references.bib}

\title{High Availability For Key-Value Stores Using Checkpoint/Restore}
\author{Fadhil Abubaker, Hussain Sadiq Abuwala}
\date{}

\begin{document}
\maketitle

\section{Introduction}

High availability (HA) is an important requirement for modern distributed
systems. HA is achieved through replication, where multiple replicas of the
system are run on different nodes for redundancy. Updates made to one replica
are propagated to the others in a synchronous or asynchronous manner.
Additionally, replication can be active-active, where any replica can accept
updates or active-passive, where only a master replica can accept updates
\cite{Dangers}. Thus, the failure of one replica does not affect the operation
of the entire system, as another replica can take its place, and continue
serving requests.

However, implementing HA within a distributed system is a challenging task. As
mentioned in \cite{RemusDB}, to build a simple active-standby replication into a
DBMS, the system has to implement propagating updates from the active replica to
the standby, coordinate transactions between the replicas and ensure atomic
handover from active to standby in the face of a failure. Moreover, these
components have to be carefully implemented so as to have minimal impact on the
performance of the underlying system.

Given the complexity of implementing HA, the question arises whether it should
be pushed outside of the system. One such approach that has been studied is
virtual machine (VM) replication, where changes made to a primary VM are
propagated to a secondary VM \cite{Hypervisor, Remus, Scales2010TheDA}. Prior
work has also looked at how to adapt datastores to run on such VMs to guarantee
HA \cite{RemusDB}.

In this paper, we design and build a key-value store that uses process-based
checkpoint/restore for active-standby replication. Our approach is similar to VM
replication, except the granularity of replication is at the process level. In a
nutshell, we capture regular snapshots of the process state and restore it

\section{Background}

\bit
  \item In-memory KV stores are used for low-latency, high throughput tasks.
  \bit
    \item Eg, web caches, ...
    \item Eg, Redis, Memcached
  \eit

  \item These KV stores implement HA through statement-based replication, where
  commands from the client are executed on the active replica and streamed to
  the standby.
  \bit
    \item For eg, Redis performs this replication asynchronously.
    \item However, in the case where the standby has disconnected and fallen
    behind significantly, Redis performs a full synchronization, where it sends
    a complete copy of the dataset to the standby. Once the standby has loaded
    the copy, the master streams commands asynchronously to it as before.
  \eit
\eit

\section{System Architecture}

\subsection{Overview}



\section{Evaluation}

For evaluating our system, we are using Yahoo Cloud Serving Benchmark (YCSB)
because it is the most popular NoSQL benchmark suite. It automates essential
benchmarking tasks such as workload definition, workload execution and
performance measurement. For YCSB to work with our key-value (KV) store, some
intermediary steps had to be followed. Firstly, we had to implement a database
interface layer. This enables the client to do actions such as "read record" and
"update record" without needing to understand our database's API. Implementing
this layer required us to code/fill out the methods specified in the abstract
class "DB". In particular, we implemented insert, update, read and delete
methods. For each method, YCSB passes in a table name and record key. As our KV
store does not have any specific table, so we just ignore that parameter. For
read methods, YCSB passes in a set of fields to be read and provides a structure
(Hash Map) to store the returned data in. For write methods, a Hash Map is
passed in comprising of a key-value pairs. Along with the aforementioned
methods, we also had to implement the "init()" function where code related to
database set up was written. Finally, the interface layer and the related
dependencies were compiled using Maven build automation tool.

\section{Improvements}

\bit
  \item Shared-disk to shared-nothing
  \item Buffer responses until replicated for consistency
  \item CRIU - apply images to process - useful for read-replicas
  \item Fault tolerance for replicator
  \item Fault tolerance for rproxy
\eit

\section{Conclusion}

\printbibliography

\end{document}
